\documentclass[11pt]{article}
\usepackage[margin=1in]{geometry}
\usepackage[T1]{fontenc}
\usepackage[utf8]{inputenc}
\usepackage{enumitem}
\usepackage{hyperref}

\title{How to create a good research project}
\author{Lukas Theißinger}
\date{\today}

\begin{document}
\maketitle

\section{Goals}
Our main goals with research projects are:
\begin{enumerate}
	\item Reproducibility
	\item Testing
	\item Usability
\end{enumerate}

The first two ensure correctness of the research and the second one is using the research, because otherwise what was the point of making it in the first place.
The question is how can we make such code bases?

\section{Project outline}
We define a project skeleton outline of files and what their purpose are and how to use them.
\begin{itemize}[leftmargin=*]
  \item \texttt{configs/}: One JSON file per experiment. These are the single source of truth for experiment parameters.
  \item \texttt{src/}: Implementation code (config loading, pipeline logic, CLI entrypoint).
  \item \texttt{tests/}: Automated tests for behavior, edge cases, and deterministic outcomes.
  \item \texttt{scripts/}: Operational scripts.
  \item \texttt{outputs/}: Generated experiment result files (not source code).
  \item \texttt{Makefile}: Unified command interface for install, test, run, and batch workflows.
\end{itemize}

How the pieces fit together:
\begin{enumerate}[leftmargin=*]
  \item Add or modify experiment configs in \texttt{configs/}.
  \item Run experiments through \texttt{make run <name>} or \texttt{make all}.
  \item Validate correctness through \texttt{make tests-all}.
  \item Use output files in \texttt{outputs/} for downstream analysis, plots, and reports.
\end{enumerate}

\section{Workflows}
We define workflows that should be intuitive and easy to use.

\subsection{How to Reproduce Results}

Use \texttt{make run} for a single experiment and \texttt{make all} for every experiment config.

\begin{enumerate}[leftmargin=*]
  \item Install dependencies once:
\begin{verbatim}
make install
\end{verbatim}

  \item Run experiments with explicit names:
\begin{verbatim}
make run experiment1
make run experiment2
\end{verbatim}
  This writes outputs to \texttt{outputs/experiment1.json} and \texttt{outputs/experiment2.json}.

  \item Run and reproduce all experiments from \texttt{configs/*.json}:
\begin{verbatim}
make run-all
\end{verbatim}
  Each config file is executed and checked for deterministic core outputs.
\end{enumerate}

If you rerun the same config and seed, the core result payload remains deterministic.

\subsection{How to Test}
\subsubsection{Running tests}
\begin{enumerate}[leftmargin=*]
  \item Run static checks:
\begin{verbatim}
make test-static
\end{verbatim}
  This runs linting and type checks.
  
  \item Run normal tests:
  \begin{verbatim}
  	make tests-pytest
  \end{verbatim}
  This runs pytest and deterministic default-config reproducibility checks.s

  \item Run deterministic reproducibility check (default config):
\begin{verbatim}
make test-deterministic
\end{verbatim}

  \item Run all tests together:
\begin{verbatim}
make tests-all
\end{verbatim}
  This runs static checks, normal tests, and deterministic checks across all configs.
\end{enumerate}

\subsubsection{Adding new tests}
\begin{enumerate}[leftmargin=*]
  \item Create tests in \texttt{tests/} and use files named \texttt{test\_*.py}.
  \item Add at least one positive path test and one failure/edge-case test for each new feature.
  \item For reproducibility-sensitive logic, include a test that asserts identical output for identical seeds/configs.
\end{enumerate}

\subsection{How to use}
The code in src/ should be written as a library to use. The experiment code should then be scripts that import stuff from that library to build results.

\end{document}
